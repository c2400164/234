\documentclass[12pt,a4paper]{article}
\usepackage[utf8]{vietnam}
\usepackage{amssymb}
\usepackage{graphicx}
\usepackage{hyperref}
\usepackage{mathtools}
\usepackage{amsthm}
\usepackage[left=2.00cm, right=2.00cm, top=2.00cm, bottom=2.00cm]{geometry}

\thispagestyle{empty}
\begin{document}
\section{Bisection Method}	
	
\subsection{Introduction}

The bisection method is a simple and effective numerical technique for finding approximate solutions to equations of the form \( f(x) = 0 \). This method relies on the Intermediate Value Theorem and requires the function to be continuous over the given interval.


\subsection{Theory}

\subsubsection{Intermediate Value Theorem}

If a function \( f(x) \) is continuous on the interval \([a, b]\) and \( f(a) \cdot f(b) < 0 \), then there is at least one point \( c \) in the interval \((a, b)\) such that \( f(c) = 0 \).

\subsubsection{Root Interval

A root interval is an interval \([a, b]\) on the real number line where a continuous function \( f(x) \) changes sign, indicating the presence of at least one root (zero) of the function within that interval. Mathematically, this is expressed as:
\[
f(a) \cdot f(b) < 0


This condition ensures that there is at least one point \( c \) in the interval \((a, b)\) such that \( f(c) = 0 \), according to the Intermediate Value Theorem.

\textbf{Key Points about Root Intervals}

\textbf{Continuity:} The function \( f(x) \) must be continuous on the interval \([a, b]\). Discontinuities can invalidate the presence of a root within the interval.

\textbf{Sign Change:} The product \( f(a) \cdot f(b) \) must be negative, indicating that the function values at \( a \) and \( b \) have opposite signs. This sign change is crucial for the existence of a root within the interval.

\textbf{Multiple Roots:} A root interval may contain more than one root. The bisection method will find one root, but additional intervals may need to be tested to locate other roots.

\textbf{Example}

Consider the function \( f(x) = x^3 - x - 2 \). To find a root interval, we evaluate the function at different points:
\( f(1) = 1^3 - 1 - 2 = -2 \)
\( f(2) = 2^3 - 2 - 2 = 4 \)

Since \( f(1) \cdot f(2) < 0 \), the interval \([1, 2]\) is a root interval for the function \( f(x) \). This means there is at least one root of  \( f(x\) within the interval \([1, 2]\).


\textbf{Practice Problems: Root Intervals}

Problem 1

Determine a root interval for the function \( f(x) = x^2 - 5x + 6 \).

Problem 2

Find a root interval for the function \( f(x) = \sin(x) - 0.5 \) in the interval \([0, 2\pi]\).

Problem 3

Identify a root interval for the function \( f(x) = e^x - 3 \) in the interval \([0, 2]\).

Problem 4

Determine a root interval for the function \( f(x) = \ln(x) - 2 \) in the interval \([1, 5]\).

Problem 5

Find a root interval for the function \( f(x) = x^3 - 4x + 1 \) in the interval \([-2, 2]\).

Problem 6

Identify a root interval for the function \( f(x) = \cos(x) - x \) in the interval \([0, 1]\).

Problem 7

Determine a root interval for the function \( f(x) = x^2 - 2 \) in the interval \([0, 2]\).

Problem 8

Find a root interval for the function \( f(x) = x^4 - 16 \) in the interval \([1, 3]\).

Problem 9

Identify a root interval for the function \( f(x) = \tan(x) - x \) in the interval \([0, \pi/2]\).

Problem 10

Determine a root interval for the function \( f(x) = x^3 - 3x + 2 \) in the interval \([-2, 2]\).



\subsubsection{Steps of the Bisection Method}

\textbf{Choose Initial Interval:} Select an interval \([a, b]\) such that \( f(a) \cdot f(b) < 0 \).

\textbf{Calculate Midpoint: }Compute the midpoint \( c \) of the interval:

\begin{align}
$x_n = \frac{a + b}{2}$
\]
\end{align}

\textbf{Check Sign:} Evaluate the function at the midpoint \( c \):

If \( f(x_n) = 0 \), then \( x_n \) is the root.

If \( f(a) \cdot f(x_n) < 0 \), then the root lies in the interval \([a, x_n]\). Set \( b = x_n \).

If \( f(b) \cdot f(x_n) < 0 \), then the root lies in the interval \([c, b]\). Set \( a = x_n \).

\textbf{Iterate:} Repeat the process until the interval \([a, b]\) is sufficiently small or \( |f(x_n)| \) is less than a predetermined tolerance level.

\subsubsection{Estimating the Number of Iterations}

To achieve a desired accuracy \( \epsilon \), the number of iterations \( n \) required can be estimated using the formula:
\[
n \geq \log_2\left(\frac{b - a}{2\epsilon}\right) = \frac{\log\left(\frac{b - a}{2\epsilon}\right)}{\log(2)} 
\]

\textbf{Proof}

To determine the number of iterations \( n \) required for the bisection method to achieve a desired accuracy \( \epsilon \), we use the following steps:

Initial Interval: Start with an interval \([a, b]\) where \( f(a) \cdot f(b) < 0 \). The length of this interval is \( L_0 = b_0 - a_0 \).

Interval Halving: At each iteration, the interval is halved. 

After the first iteration, the interval length is:
\[
L_1 = \frac{L_0}{2}
\]
After the second iteration, the interval length is:
\[
L_2 = \frac{L_1}{2} = \frac{L_0}{2^2}
\]
After \( n \) iterations, the interval length is:
\[
b_n-a_n = L_n =   \frac{L_0}{2^n} = \left(\dfrac{1}{2}\right)^n (b_0 - a_0)
\]
Desired Accuracy: We want the interval length \( L_n \) to be less than or equal to the desired accuracy \( \epsilon \):
\[
|x^* - x_n| \leq \dfrac{1}{2}(b_n-a_n) = \dfrac{1}{2}\left(\dfrac{1}{2}\right)^n (b_0 - a_0) = \left(\dfrac{1}{2}\right)^{n+1} (b_0 - a_0)
\]
Final Formula:
\[
n \geq \log_2\left(\frac{b - a}{2\epsilon}\right)
\]
This formula tells us the minimum number of iterations \( n \) required to reduce the interval length to within the desired accuracy \( \epsilon \).

\subsubsection{Example}

Example 1

Find an approximate root of the equation \( f(x) = x^3 - x - 2 \) in the interval \([1, 2]\).

Choose Initial Interval: \( a = 1 \), \( b = 2 \)
\[
f(1) = -2, \quad f(2) = 4
\]

Calculate Midpoint: \( c = \frac{1 + 2}{2} = 1.5 \)

Check Sign: \( f(1.5) = -0.125 \)

Since \( f(1) \cdot f(1.5) < 0 \), set \( b = 1.5 \).

Iterate: Continue the process with the new interval \([1, 1.5]\).



6. Practice Problems

Problem 1

Use the bisection method to find an approximate root of the equation \( f(x) = \cos(x) - x \) in the interval \([0, 1]\) with an accuracy of \( \epsilon = 0.01 \).

Problem 2

Apply the bisection method to find the root of the equation \( f(x) = x^2 - 4 \) in the interval \([1, 3]\) with an accuracy of \( \epsilon = 0.001 \).

Problem 3

Find the root of the equation \( f(x) = e^x - 3x \) using the bisection method in the interval \([0, 1]\) with an accuracy of \( \epsilon = 0.0001 \).

Problem 4

Determine the root of the equation \( f(x) = \ln(x) + x^2 - 3 \) in the interval \([1, 2]\) using the bisection method with an accuracy of \( \epsilon = 0.0005 \).

Problem 5

Use the bisection method to find the root of the equation \( f(x) = x^3 - 2x^2 + x - 1 \) in the interval \([0, 2]\) with an accuracy of \( \epsilon = 0.01 \).


\section{Đề toán}
\begin{enumerate}[\bf Câu 1.]
	\item Giải các phương trình - bất phương trình sau\\
a) $\sqrt{4 x^2+8 x-37}=\sqrt{-x^2-2 x+3}$\\
b)	$\sqrt{-x^2+3 x+1}=x-4$\\
c) $(x-3) \sqrt{x^2+4}=x^2-9$\\
d) $x^2+\sqrt{x+1}=1$\\
e) $\sqrt{x+9}=5-\sqrt{2x+4}$\\
f) $\sqrt{5 x^2+10 x+1}=7-x^2-2 x$\\
g) $\dfrac{1}{x^2-x+1} \leq \dfrac{1}{2 x^2+x+2}$\\
h)	$x^4-3 x^2+2 \leq 0$
\item Tìm các giá trị của tham số $m$ để:\\
a) Hàm số $y=\dfrac{1}{\sqrt{m x^2-2 m x+5}}$ có tập xác định $\mathbb{R}$.\\
b) Tam thức bậc hai $y=-x^2+m x-1$ có dấu không phụ thuộc vào $x$.\\
c) Hàm số $y=\sqrt{-2 x^2+m x-m-6}$ có tập xác định chỉ gồm một phần tử.

\item Cho đường thẳng $d: x-3 y+1=0$ và điềm $M(-1 ;-4)$.\\
a) Tìm tọa độ điểm $N$ thuộc $d$ sao cho $MN = 4.$\\
b) Tìm tọa độ điểm $M'$ đối xứng với $M$ qua $d$.\\
c) Tìm tọa độ trực tâm $H$ của $\Delta MM'N$.
\item Cho hình bình hành $A B C D$ có phương trình đường thẳng chứa cạnh $A B, A D$ lần lượt là $2 x-y-3=0, x-2 y+1=0$ và $C(3 ; 0)$. Tìm tọa độ các đỉnh và phương trình các cạnh còn lại của hình bình hành $A B C D$.
\end{enumerate}



\end{document}