\documentclass[12pt,a4paper,two side]{article}
\usepackage[ left=2.00cm, right=2.00cm, top=2.00cm, bottom=2.00cm]{geometry}
\usepackage[utf8]{vietnam}
\usepackage{amsmath,amsfonts,amssymb,amsthm}
\usepackage{pgf,tikz}
\usepackage{mathrsfs}
\usetikzlibrary{arrows}
\usepackage{enumerate}
\usepackage{tcolorbox}
\usepackage{multicol}
\usepackage{color}
\usepackage{tcolorbox}
\usepackage{hyperref}
%\hypersetup{ pdfborder={0 0 0}  % Tắt viền liên kết}
\usepackage{xcolor}

\definecolor{Darkblue}{rgb}{0, 0, 0.545}

\usepackage{tocloft}


\usepackage{cite}
\usepackage[backend=biber,sorting=none]{biblatex}
\addbibresource{references.bib}



\addbibresource{references.bib}

\begin{document}

\begin{titlepage}
		\centering
		Trường Đại học Tôn Đức Thắng\\
		Khoa Toán – Thống kê\\[2cm]
\begin{figure}[h]
	\centering
	\includegraphics[width=0.4\linewidth]{Logo ĐH Tôn Đức Thắng-TDT.png}
	
	\label{fig:logo-h-ton-uc-thang-tdt}
\end{figure}

	

		
	
	
		\huge
		TIỂU LUẬN GIỮA KỲ\\[2cm]
		\normalsize
		
		Năm học: 2024 - 2025 Học kỳ: 2\\[2cm]
		Môn học: Soạn thảo tài liệu khoa học bằng \LaTeX\\
		Mã môn học: C02045\\[2cm]
		Họ và tên sinh viên thực hiện: Trần Tấn Tài\\
		MSSV: C2400164\\[2cm]
		Mã đề: 43\\[2cm]
		Giảng viên giảng dạy: TS. Nguyễn Hữu Cần\\
		\vfill
		TP. HCM, ngày 05 tháng 03 năm 2025
	\end{titlepage}
	
	\section*{Lời cam kết}
	\begin{enumerate}[]
		\item Tiểu luận này được tôi biên soạn lại bằng phần mềm \LaTeX với mục đích hoàn
		thành môn học thực hành Soạn thảo tài liệu khoa học bằng \LaTeX, ngoài ra
		không còn mục đích nào khác.
		\item Tiểu luận này được biên soạn bằng \LaTeX dựa trên tài liệu sau:
		\begin{enumerate}[]
			\item Tiêu đề: An application of fixed point theorem to best approximation in locally
			convex space
			mixed type
			\item Tác giả: Hemant Kumar Nashine, Mohammad Saeed Khanb
		\end{enumerate}
	\end{enumerate}
	
	\begin{flushright}
		\begin{minipage}{.3\textwidth}
			\centering 
			Sinh viên thực hiện\\[1cm]
			\textit{(ký tên)}\\[1cm]
			Trần Tấn Tài
		\end{minipage}
	\end{flushright}
	\newpage
	

{\LARGE\textbf{An application of fixed point theorem to best approximation in locally convex space}}\\

\vspace{0.5cm}

\fontsize{15pt}{14pt}{\textbf{Hemant Kumar Nashine ${ }^{\hyperlink{phana}{\text {\textcolor{Darkblue}{a} }}}$}, \textbf{Mohammad Saeed Khan ${ }^{\hyperlink{phanb}{\text {\textcolor{Darkblue}{b,*} }}}$}}\\

\hypertarget{phana}{${ }^{\text {a }}$ \fontsize{12pt}{12pt}\textit{Department of Mathematics, Disha Institute of Management and Technology, Satya Vihar, Vidhansabha-Chandrakhuri Marg, Nardha, Mandir Hasaud, Raipur-492101 (Chhattisgarh), India}}\\
\hypertarget{phanb}{${ }^{\text {b }}$} \fontsize{12pt}{12pt}\textit{Department of Mathematics and Statistics, Sultan Qaboos University, P.O. Box 36, PCode 123 Al-Khod, Muscat, Sultanate of Oman, Oman}\\

\vspace{0.7cm}

\textbf{ 1. Introduction}\\

During the last four decades several interesting and valuable results were studied extensively in the field of fixed point
 theorems.\\
 In 1990, Jungck \cite{1}  obtained the following theorem for compatible mapping:\\

 \hypertarget{muc1.1}{\textbf{ Theorem 1.1}}(\cite{citation-key1}). Let $\mathcal{T}$ and $\ell$ be compatible self-maps of a closed convex subset $\mathcal{M}$ of a Banach space $X$. Suppose $\ell$ is linear, continuous, and that $\mathcal{T}(\mathcal{M}) \subseteq \ell(\mathcal{M})$. If there exists $a \in(0,1)$ such that $x, y \in \mathcal{M}$

$$
\|\mathcal{T} x-\mathcal{T} y\| \leq a\|\ell x-\ell y\|+(1-a) \max \{\|\mathcal{T} x-\ell x\|,\|\mathcal{T} y-\ell y\|\}
$$
then $\mathcal{T}$ and $\ell$ have a unique common fixed point in $\mathcal{M}$.\\

In this paper, we first derive a common fixed point result in locally convex space which generalizes the result of Jungck \cite{citation-key1}. This new result is used to prove another fixed point result for best approximation. By doing so, we in fact, extend and improve the result of Brosowski \cite{citation-key2}, Meinardus \cite{citation-key3}, Sahab et al. \cite{citation-key4}, Singh \cite{citation-key5,citation-key6,citation-key7} and many others.\\

\textbf{2. Preliminaries}\\

In the material to be presented here, the following definitions have been used:\\

 In what follows, $(\varepsilon, \tau)$ will be a Hausdorff locally convex topological vector space. A family $\left\{p_\alpha: \alpha \in \Delta\right\}$ of seminorms defined on $\mathcal{E}$ is said to be an associated family of seminorms for $\tau$ if the family $\{\gamma U: \gamma>0\}$, where $U=\bigcap_{i=1}^n \mathcal{U}_{\alpha_i}, n \in \mathbb{N}$, and $U_{\alpha_i}=\left\{x \in \mathcal{E}: p_{\alpha_i}(x) \leq 1\right\}$, forms a base of neighbourhoods of zero for $\tau$. A family $\left\{p_\alpha: \alpha \in \Delta\right\}$ of seminorms defined on $\mathcal{E}$ is called an augmented associated family for $\tau$ if $\left\{p_\alpha: \alpha \in \Delta\right\}$ is an associated family with the property thatthe seminorm $\max \left\{p_\alpha, p_\beta\right\} \in\left\{p_\alpha: \alpha \in \Delta\right\}$ for any $\alpha, \beta \in \Delta$. The associated and augmented families of seminorms will be denoted by $\mathcal{A}(\tau)$ and $\mathcal{A}^*(\tau)$, respectively. It is well known that given a locally convex space $(\mathcal{E}, \tau)$, there always exists a family $\left\{p_\alpha: \alpha \in \Delta\right\}$ of seminorms defined of $\mathcal{E}$ such that $\left\{p_\alpha: \alpha \in \Delta\right\}=\mathcal{A}^*(\tau)$ (see [\cite{citation-key8}, pp 203]). A subset $\mathcal{M}$ of $\mathcal{E}$ is $\tau$-bounded if and only if each $p_\alpha$ is bounded on $\mathcal{M}$.\\
\newpage
 * Corresponding author.\\
E-mail addresses:\\
nashine\_09@rediffmail.com, hemantnashine@gmail.com 
(H.K. Nashine),\\
mohammad@squedu.om (M.S. Khan).\\
0893-9659/\$ - see front matter © 2009 Published by Elsevier Ltd\\
doi: 10.1016/j.aml.2009.06.025\\

 Suppose that $\mathcal{M}$ is a $\tau$-bounded subset of $\mathcal{E}$. For this set $\mathcal{M}$, we can select a number $\lambda_\alpha>0$ for each $\alpha \in \Delta$ such that $\mathcal{M} \subset \lambda_\alpha U_\alpha$ where $U_\alpha=\left\{x \in \mathcal{M}: p_\alpha(x) \leq 1\right\}$. Clearly, $\mathscr{B}=\bigcap_\alpha \lambda_\alpha U_\alpha$ is $\tau$-bounded, $\tau$-closed, absolutely convex and contains $\mathcal{M}$. The linear span $\varepsilon_{\mathscr{B}}$ of $\mathscr{B}$ in $\mathcal{\varepsilon}$ is $\bigcup_{n=1}^{\infty} n \mathscr{B}$. The Minkowski functional of $\mathscr{B}$ is a norm $\|\cdot\|_{\mathcal{B}}$ on $\varepsilon_{\mathscr{B}}$. Thus, $\left(\varepsilon_{\mathcal{B}},\|\cdot\|_{\mathcal{B}}\right)$ is a normed space with $\mathscr{B}$ as its closed unit ball and $\sup _\alpha p_\alpha\left(x / \lambda_\alpha\right)=\|x\|_{\mathcal{B}}$ for each $x \in \mathcal{E}_{\mathscr{B}}$. 
 (for details, see  \cite{citation-key9,citation-key10,citation-key11}).\\

\textbf{Definition 2.1} (\cite{citation-key9}). Let $\ell$ and $\mathcal{T}$ be self-maps on $\mathcal{M}$. The map $\mathcal{T}$ is called\\
(i) $\mathcal{A}^*(\tau)$-nonexpansive if for all $x, y \in \mathcal{M}$

$$
p_\alpha(\mathcal{T} x-\mathcal{T} y) \leq p_\alpha(x-y)
$$

for each $p_\alpha \in \mathcal{A}^*(\tau)$.
(ii) $\mathcal{A}^*(\tau)-\ell$-nonexpansive if for all $x, y \in \mathcal{M}$

$$
p_\alpha(\mathcal{T} x-\mathcal{T} y) \leq p_\alpha(\ell x-\ell y)
$$

for each $p_\alpha \in \mathcal{A}^*(\tau)$.\\

For simplicity, we shall call $\mathscr{A}^*(\tau)$-nonexpansive $\left(\mathcal{A}^*(\tau)-\ell\right.$-nonexpansive) maps to be nonexpansive ( $\ell$-nonexpansive).\\

\textbf{Definition 2.2} (\cite{citation-key11}). A pair of self-mappings $(\mathcal{T}, \ell)$ of a locally convex space $(\mathcal{E}, \tau)$ is said to be compatible, if $p_\alpha\left(\mathcal{T} \ell x_n-\right.$ $\left.\ell \mathcal{T} x_n\right) \rightarrow 0$, whenever $\left\{x_n\right\}$ is a sequence in $\mathcal{E}$ such that $\mathcal{T} x_n, \ell x_n \rightarrow t \in \mathcal{E}$.\\
Every commuting pair of mappings is compatible but the converse is not true in general.\\

\textbf{Definition 2.3} Suppose that $\mathcal{M}$ is $q$-starshaped with $q \in \mathcal{F}(\ell)$ and is both $\mathcal{T}$ - and $\ell$-invariant. Then $\mathcal{T}$ and $\ell$ are called $\mathcal{R}$-subcommuting \cite{citation-key12, citation-key13,citation-key14} on $\mathcal{M}$, if for all $x \in \mathcal{M}$ and for all $p_\alpha \in \mathcal{A}^*(\tau)$, there exists a real number $\mathcal{R}>0$ such that $p_\alpha(\ell \mathcal{T} x-\mathcal{T} \ell x) \leq\left(\frac{\mathcal{R}}{k}\right) p_\alpha(((1-k) q+k \mathcal{T} x)-\ell x)$ for each $k \in(0,1)$. If $\mathcal{R}=1$, then the maps are called 1 -subcommuting. The $\ell$ and $\mathcal{T}$ are called $\mathscr{R}$-subweakly commuting \cite{citation-key15} on $\mathcal{M}$, if for all $x \in \mathcal{M}$ and for all $p_\alpha \in \mathcal{A}^*(\tau)$, there exists a real number $\mathcal{R}>0$ such that $p_\alpha(\ell \mathcal{T} x-\mathcal{T} \ell x) \leq \mathcal{R} d_{p_\alpha}(\ell x,[q, \mathcal{T} x])$, where $[q, x]=(1-k) q+k x: 0 \leq k \leq 1$.\\

\textbf{Remark 2.4}. (1) It is obvious that commutativity implies $\mathscr{R}$-subcommutativity, which in turn implies $\mathscr{R}$-weakly commutativity \cite{ citation-key13} , \cite{citation-key14}.
(2) It is also well known that commuting maps are $\mathscr{R}$-subweakly commuting maps and $\mathscr{R}$-subweakly commuting maps are $\mathscr{R}$-weakly commuting but not conversely in general (see \cite{citation-key15}).

To clear the above remarks, in the following, we have furnished some examples:\\

\textbf{Example 2.5.} Let $\mathcal{X}=\mathbb{R}$ with norm $\|x\|=|x|$ and $\mathcal{M}=[1, \infty)$. Let $\mathcal{T}, s: \mathcal{M} \rightarrow \mathcal{M}$ be defined by

$$
\mathcal{T} x=x^2 \quad \text { and } \quad s x=2 x-1
$$

for all $x \in \mathcal{M}$. Then $\mathcal{T}$ and $s$ are $\mathcal{R}$-weakly commuting with $\mathcal{R}=2$. However, they are not $\mathscr{R}$-subcommuting because

$$
|\mathcal{T} s x-s \mathcal{T} x| \leq\left(\frac{\mathcal{R}}{k}\right)|(k \mathcal{T} x+(1-k) p)-s x|
$$

does not hold for $x=2$ and $k=\frac{2}{3}$, where $p=1 \in \mathscr{F}(s)$.\\

\textbf{Example 2.6.} Let $\mathcal{X}=\mathbb{R}$ with norm $\|x\|=|x|$ and $\mathcal{M}=[1, \infty)$. Let $\mathcal{T}, s: \mathcal{M} \rightarrow \mathcal{M}$ be defined by

$$
\mathcal{T} x=4 x-3 \quad \text { and } \quad s x=2 x^2-1
$$

for all $x \in \mathcal{M}$. Then $\mathcal{M}$ is $p$-starshaped with $p=1 \in \mathcal{F}(\delta)$ and is both $\mathcal{T}$ and $s$-invariant. Also, $|\mathcal{T} \delta x-s \mathcal{T} x|=24(x-1)^2$. Further,

$$
|\mathcal{T} s x-s \mathcal{T} x| \leq\left(\frac{\mathcal{R}}{k}\right)|(k \mathcal{T} x+(1-k) p)-s x|
$$

for all $x \in \mathcal{M}$, where $\mathcal{R}=12$ and $p=1 \in \mathcal{F}(s)$. Thus, $\mathcal{T}$ and $s$ are $\mathcal{R}$-subcommuting on $\mathcal{M}$ but are not commuting on $\mathcal{M}$.\\

\textbf{Example 2.7.} Let $\mathcal{X}=\mathbb{R}^2$ with norm $\|(x, y)\|=\max \{|x|,|y|\}$, and let $\mathcal{T}$ and $s$ be defined by

$$
\mathcal{T}(x, y)=\left(2 x-1, y^3\right) \quad \text { and } \quad s(x, y)=\left(x^2, y^2\right)
$$
for all $(x, y) \in \mathcal{X}$. Then $\mathcal{T}$ and $s$ are $\mathcal{R}$-subweakly commuting on $\mathcal{M}=\{(x, y): x \geq 1, y \geq 1\}$ but they are not commuting on $\mathcal{M}$.\\

\textbf{Definition 2.8.} Suppose that $\mathcal{M}$ is $q$-starshaped with $q \in \mathcal{F}(\ell)$.\\
Define $\bigwedge_q(\ell, \mathcal{T})=\left\{\wedge\left(\ell, \mathcal{J}_k\right): 0 \leq k \leq 1\right\}$ where $\mathcal{T}_k x=(1-k) q+k \mathcal{T} x$ and\\
$\bigwedge\left(\ell, \mathcal{T}_k\right)=\left\{\left\{x_n\right\} \subset \mathcal{M}:\lim _n \ell x_n=\lim _n \mathcal{T}_k x_n=t \in \mathcal{M} \Rightarrow \lim _n p_\alpha\left(\ell \mathcal{T}_k x_n-\mathcal{T}_k \ell x_n\right)=0\right\}$, for all sequences $\left\{x_n\right\} \in \bigwedge_q(\ell, \mathcal{T})$. Then $\ell$ and $\mathcal{T}$ are called subcompatible \cite{citation-key16},\cite{17} if

$$
\lim _n p_\alpha\left(\mathfrak{J} x_n-\mathcal{T} \ell x_n\right)=0
$$
for all sequences $x_n \in \bigwedge_q(\ell, \mathcal{T})$.\\

Obviously, subcompatible maps are compatible but the converse does not hold, in general, as the following example shows.\\

\textbf{Example 2.9.} Let $\mathcal{X}=\mathbb{R}$ with usual norm and $\mathcal{M}=[1, \infty)$. Let $\ell(x)=2 x-1$ and $\mathcal{T}(x)=x^2$, for all $x \in \mathcal{M}$. Let $q=1$. Then $\mathcal{M}$ is $q$-starshaped with $\ell q=q$. Note that $\ell$ and $\mathcal{T}$ are compatible. For any sequence $\left\{x_n\right\}$ in $\mathcal{M}$ with $\lim _n x_n=2$, we have, $\lim _n \ell x_n=\lim _n \mathcal{T}_{\frac{2}{3}} x_n=3 \in \mathcal{M} \Rightarrow \lim _n\left\|\ell \mathcal{T}_{\frac{2}{3}} x_n-\mathcal{T}_{\frac{2}{3}} \ell x_n\right\|=0$. However, $\lim _n\left\|\ell \mathcal{T} x_n-\mathcal{T} \ell x_n\right\| \neq 0$. Thus $\ell$ and $\mathcal{T}$ are not subcompatible maps.\\

Note that $\mathscr{R}$-subweakly commuting and $\mathcal{R}$-subcommuting maps are subcompatible. The following simple example reveals that the converse is not true, in general.\\

\textbf{Example 2.10.} Let $\mathcal{X}=\mathbb{R}$ with usual norm and $\mathcal{M}=[0, \infty)$. Let $\ell(x)=\frac{x}{2}$ if $0 \leq x<1$ and $\ell x=x$ if $x \geq 1$, and $\mathcal{T}(x)=\frac{1}{2}$ if $0 \leq x<1$ and $\mathcal{T} x=x^2$ if $x \geq 1$. Then $\mathcal{M}$ is 1 -starshaped with $\ell 1=1$ and $\bigwedge_q(\ell, \mathcal{T})=\left\{\left\{x_n\right\}: 1 \leq x_n<\infty\right\}$. Note that $\ell$ and $\mathcal{T}$ are subcompatible but not $\mathcal{R}$-weakly commuting for all $\mathcal{R}>0$. Thus $\ell$ and $\mathcal{T}$ are neither $\mathscr{R}$-subweakly commuting nor $\mathcal{R}$-subcommuting maps.\\

\textbf{Definition 2.11} (\cite{citation-key9}). Let $x_0 \in \mathcal{E}$ and $\mathcal{M} \subseteq \mathcal{E}$. Then for $0<a \leq 1$, we define the set $\mathscr{D}_a$ of best ( $\left.\mathcal{M}, a\right)$-approximant to $x_0$ as follows:

$$
\mathscr{D}_a=\left\{y \in \mathcal{M}: a p_\alpha\left(y-x_0\right)=d_{p_\alpha}\left(x_0, \mathcal{M}\right), \quad \text { for all } p_\alpha \in \mathcal{A}^*(\tau)\right\}
$$

where

$$
d_{p_\alpha}\left(x_0, \mathcal{M}\right)=\inf \left\{p_\alpha\left(x_0-z\right): z \in \mathcal{M}\right\}
$$
For $a=1$, definition reduces to the set $\mathscr{D}$ of best $\mathcal{M}$-approximant to $x_0$.\\

\textbf{Definition 2.12.} The map $\mathcal{T}: \mathcal{M} \rightarrow \mathcal{E}$ is said to be demiclosed at 0 if for every net $\left\{x_n\right\}$ in $\mathcal{M}$ converging weakly to $x$ and $\left\{\mathcal{T} x_n\right\}$ converging strongly to 0 , we have $\mathcal{T} x=0$.

Throughout, this paper $\mathcal{F}(\mathcal{T})$ (resp. $\mathcal{F}(\ell))$ denotes the fixed point set of mapping $\mathcal{T}$ (resp. $(\ell)$ ).\\


\textbf{3. Main result}\\

To prove the main result, a lemma is presented below:\\

\hypertarget{muc3.1}{\textbf{Lemma 3.1.}} Let $\mathcal{T}$ and $\ell$ be compatible self-maps of a $\tau$-bounded subset $\mathcal{M}$ of a Hausdorff locally convex space $(\mathcal{E}, \tau)$. Then $\mathcal{T}$ and $\ell$ be compatible on $\mathcal{M}$ with respect to $\|\cdot\|_{\mathcal{B}}$.\\

\textbf{Proof.} By hypothesis for each $p_\alpha \in \mathcal{A}^*(\tau)$,

$$
p_\alpha\left(\mathcal{T} \ell x_n-\ell \mathcal{T} x_n\right) \rightarrow 0,  \ \ \ \ \ \ \ \ \ \ \  \ \ \ \ \ \ \ \ \ \ \ (3.1)
$$ 

whenever $\left\{x_n\right\}$ is a sequence in $\mathcal{M}$ such that

$$
p_\alpha\left(\mathcal{T} x_n-t\right) \rightarrow 0, \quad p_\alpha\left(\ell x_n-t\right) \rightarrow 0
$$

for some $t \in \mathcal{M}$.
Taking supremum on both sides,

$$
\sup _\alpha p_\alpha\left(\frac{\mathcal{T} \ell x_n-\ell \mathcal{T} x_n}{\lambda_\alpha}\right) \rightarrow 0
$$

\textbf{i.e.,}

$$
\left\|\mathcal{T} \ell x_n-\ell \mathcal{T} x_n\right\|_{\mathcal{B}} \rightarrow 0
$$

whenever $\left\{x_n\right\}$ is a sequence in $\mathcal{M}$ such that

$$
\sup _\alpha p_\alpha\left(\frac{\mathcal{T} x_n-t}{\lambda_\alpha}\right) \rightarrow 0, \quad \sup _\alpha p_\alpha\left(\frac{\ell x_n-t}{\lambda_\alpha}\right) \rightarrow 0
$$
\textbf{i.e.,}

$$
\left\|\mathcal{T} x_n-t\right\|_{\mathscr{B}} \rightarrow 0, \quad\left\|\ell x_n-t\right\|_{\mathcal{B}} \rightarrow 0
$$


A technique of Tarafdar \cite{citation-key10} to obtain the following common fixed point theorem which generalizes\hyperlink{muc1.1}{{\textcolor{Darkblue}{Theorem 1.1.}}}\\

\textbf{Theorem 3.2.} Let $\mathcal{M}$ be a nonempty $\tau$-bounded, $\tau$-sequentially complete and convex subset of a Hausdorff locally convex space $(\mathcal{E}, \tau)$. Let $\mathcal{T}$ and $\ell$ be compatible self-maps of $\mathcal{M}$ such that $\mathcal{T}(\mathcal{X}) \subseteq \ell(\mathcal{X}), \ell$ is linear and nonexpansive, and satisfying

$$
p_\alpha(\mathcal{T} x-\mathcal{T} y) \leq a p_\alpha(\ell x-\ell y)+(1-a) \max \left\{p_\alpha(\mathcal{T} x-\ell x), p_\alpha(\mathcal{T} y-\ell y)\right\} \ \ \ \ \ \ \ \ \ \ \  \ \ \  \hypertarget{dau3.2}{(3.2)}
$$
for all $x, y \in \mathcal{M}$ and $p_\alpha \in \mathcal{A}^*(\tau)$, and for some $a \in(0,1)$, then $\mathcal{T}$ and $\ell$ have a unique common fixed point.\\

\textbf{Proof.} Since the norm topology on $\varepsilon_{\mathcal{B}}$ has a base of neighbourhoods of zero consisting of $\tau$-closed sets and $\mathcal{M}$ is $\tau$ sequentially complete, therefore, $\mathcal{M}$ is a $\|\cdot\|_B$-sequentially complete subset of ( $\mathcal{E}_{\mathscr{B}},\|\cdot\|_{\mathcal{B}}$ ) (Theorem 1.2, \cite{citation-key10}). \hyperlink{muc3.1}{\textcolor{Darkblue}{By Lemma 3.1}}, $\mathcal{T}$ and $\ell$ are $\|\cdot\|_{\mathcal{B}}$-compatible maps of $\mathcal{M}$. From \hyperlink{dau3.2}{\textcolor{Darkblue}{(3.2)}}, we obtain for $x, y \in \mathcal{M}$,

$$
\sup _\alpha p_\alpha\left(\frac{\mathcal{T} x-\mathcal{T} y}{\lambda_\alpha}\right) \leq a \sup _\alpha p_\alpha\left(\frac{\ell x-\ell y}{\lambda_\alpha}\right)+(1-a) \max \left\{\sup _\alpha p_\alpha\left(\frac{\mathcal{T} x-\ell x}{\lambda_\alpha}\right), \sup _\alpha p_\alpha\left(\frac{\mathcal{T} y-\ell y}{\lambda_\alpha}\right)\right\}
$$


Thus

$$
\|\mathcal{T} x-\mathcal{T} y\|_{\mathcal{B}} \leq a\|\ell x-\ell y\|_{\mathcal{B}}+(1-a) \max \left\{\|\mathcal{T} x-\ell x\|_{\mathcal{B}},\|\mathcal{T} y-\ell y\|_{\mathcal{B}}\right\}
\ \ \ \ \ \ \ \ \ \ \  \ \ \ (3.3)$$


Note that, if $\ell$ is nonexpansive on a $\tau$-bounded, $\tau$-sequentially complete subset $\mathcal{M}$ of $\mathcal{E}$, then $\ell$ is also nonexpansive with respect to $\|\cdot\|_{\mathcal{B}}$ and hence $\|\cdot\|_{\mathcal{B}}$-continuous \cite{citation-key8}. A comparison of our hypothesis with that of \hyperlink{muc1.1}{\textcolor{Darkblue}{Theorem 1.1}} tells that we can apply\hyperlink{muc1.1}{ \textcolor{Darkblue}{Theorem 1.1}} to $\mathcal{M}$ as a subset of $\left(\mathcal{E}_{\mathcal{B}},\|\cdot\|_{\mathcal{B}}\right)$ to conclude that there exists a unique $w \in \mathcal{M}$ such that $w=\mathcal{T} w=\ell w$.\\

\textbf{Example 3.3.} Let $X=\mathbb{R}$ with usual norm and $\mathcal{M}=[0,1]$. Let $\mathcal{T}(x)=1$ for $0 \leq x \leq \dfrac{1}{2}$, and $\mathcal{T}(x)=0$ for $\dfrac{1}{2}<x \leq 1, \ell(x)=0$ for $0<x \leq \dfrac{1}{2}$, and $\ell(x)=1$ for $\dfrac{1}{2}<x \leq 1$. Then all the assumptions of Theorem 3.2 are satisfied, but $\mathcal{T}$ and $\ell$ have no common fixed point.\\

\hypertarget{muc3.4}{\textbf{Theorem 3.4.}} Let $\mathcal{M}$ be a nonempty $\tau$-bounded, $\tau$-sequentially complete and convex subset of a Hausdorff locally convex space $(\varepsilon, \tau)$. Let $\mathcal{T}$ and $\ell$ be self-maps of $\mathcal{M}$ such that $\mathcal{T}$ and $\ell$ are subcompatible. Suppose that $\mathcal{T}$ and $\ell$ satisfy \hyperlink{dau3.2}{\textcolor{Darkblue}{(3.2)}}, $\ell$ is linear and nonexpansive, $\ell(\mathcal{M})=\mathcal{M}, q \in \mathcal{F}(\ell)$, then $\mathcal{T}$ and $\Omega$ have a common fixed point provided one of the following conditions holds:\\
(i) $\mathcal{M}$ is $\tau$-sequentially compact and $\mathcal{T}$ is continuous;\\
(ii) $\mathcal{T}$ is a compact map;\\
(iii) $\mathcal{M}$ is weakly compact in $(\mathcal{\ell}, \tau), \ell$ is weakly continuous and $\ell-\mathcal{T}$ is demiclosed at 0 .\\

\textbf{Proof.} Choose a monotonically nondecreasing sequence $\left\{k_n\right\}$ of real numbers such that $0<k_n<1$ and $\lim \sup k_n=1$. For each $n \in \mathbb{N}$, define $\mathcal{T}_n: \mathcal{M} \rightarrow \mathcal{M}$ as follows:
$$
\mathcal{T}_n x=k_n \mathcal{T} x+\left(1-k_n\right) q \ \ \ \ \ \ \ \ \ \ \ \ \ \ \ \ \ \hypertarget{dau3.4}{(3.4)} 
$$
Obviously, for each $n, \mathcal{T}_n$ maps $\mathcal{M}$ into itself, since $\mathcal{M}$ is convex.
As $\ell$ is linear, we can have\\
$$
\mathcal{T}_m \ell x_n=k_n \mathcal{T} \ell x_n+\left(1-k_n\right) q
$$
and
$$
l \mathcal{T}_m x=k_n \ell \mathcal{T} x_n+\left(1-k_n\right) \ell q
$$
The subcompatibility of $\ell$ and $\mathcal{T}$ and $q \in \mathcal{F}(\ell)$ implies that
$$
\begin{aligned}
0 & \leq \lim _n p_\alpha\left(\mathcal{T}_n \ell x_m-\ell \mathcal{T}_n x_m\right) \\
& \leq \lim _m k_n p_\alpha\left(\mathcal{T} \ell x_m-\ell \mathcal{T} x_m\right)+\lim _m\left(1-k_n\right) p_\alpha(q-\ell q) \\
& =0
\end{aligned}
$$

for any $\left\{x_m\right\} \subset \mathcal{M}$ with $\lim _m \mathcal{T}_n x_m=\lim _m \ell x_m=t \in \mathcal{M}$.\\
Hence $\left\{\mathcal{T}_n\right\}$ and $\ell$ are compatible for each $n$ and $x_n \in \mathcal{M}$ and $\mathcal{T}_n(\mathcal{M}) \subseteq \mathcal{M}=\ell(\mathcal{M}), \boldsymbol{l}$ is linear and $q \in \mathcal{F}(\ell)$. Therefore $\mathcal{T}_n(\mathcal{M}) \subseteq \ell(\mathcal{M})$.\\

For all $x, y \in \mathcal{M}, p_\alpha \in \mathcal{A}^*(\tau)$ and for all $j \geq n$, ( $n$ fixed), we obtain from \hyperlink{dau3.2}{\textcolor{Darkblue}{(3.2)}} and \hyperlink{dau3.4}{\textcolor{Darkblue}{(3.4)}} that\\

$$
\begin{aligned}
p_\alpha\left(\mathcal{T}_n x-\mathcal{T}_n y\right)= & k_n p_\alpha(\mathcal{T} x-\mathcal{T} y) \leq k_j p_\alpha(\mathcal{T} x-\mathcal{T} y) \\
\leq & p_\alpha(\mathcal{T} x-\mathcal{T} y) \\
\leq & a p_\alpha(\ell x-\ell y)+(1-a) \max \left\{p_\alpha(\mathcal{T} x-I x), p_\alpha(\mathcal{T} y-\ell y)\right\} \\
\leq & a p_\alpha(\ell x-\ell y)+(1-a) \max \left\{p_\alpha\left(\mathcal{T} x-\mathcal{T}_n x\right)+p_\alpha\left(\mathcal{T}_n x-\ell x\right), p_\alpha\left(\mathcal{T} y-\mathcal{T}_n y\right)+p_\alpha\left(\mathcal{T}_n y-\ell y\right)\right\} \\
\leq & a p_\alpha(\ell x-\ell y)+(1-a) \max \left\{\left(1-k_n\right) p_\alpha(\mathcal{T} x-q)\right. \\
& \left.+p_\alpha\left(\mathcal{T}_n x-\ell x\right),\left(1-k_n\right) p_\alpha(\mathcal{T} y-q)+p_\alpha\left(\mathcal{T}_n y-\ell y\right)\right\}
\end{aligned}
$$


Hence for all $j \geq n$, we have\\
$$
\begin{aligned}
p_\alpha\left(\mathcal{T}_n x-\mathcal{T}_n y\right) \leq & a p_\alpha(\ell x-\ell y)+(1-a) \max \left\{\left(1-k_j\right) p_\alpha(\mathcal{T} x-q)\right. \\
& \left.+p_\alpha\left(\mathcal{T}_n x-\ell x\right),\left(1-k_j\right) p_\alpha(\mathcal{T} y-q)+p_\alpha\left(\mathcal{T}_n y-\ell y\right)\right\}   \ \ \ \ \ \ \ \ \ \  \ \ \ \ \ \ \ \ \ \ (3.5)
\end{aligned}
$$

As $\lim k_j=1$, from (3.5), for every $n \in \mathbb{N}$, we have
$$
\begin{aligned}
p_\alpha\left(\mathcal{T}_n x-\mathcal{T}_n y\right)= & \lim _j p_\alpha\left(\mathcal{T}_n x-\mathcal{T}_n y\right) \\
\leq & \lim _j\left\{a p_\alpha(\ell x-\ell y)+(1-a) \max \left\{\left(1-k_j\right) p_\alpha(\mathcal{T} x-q)\right.\right. \\
& \left.\left.+p_\alpha\left(\mathcal{T}_n x-\ell x\right),\left(1-k_j\right) p_\alpha(\mathcal{T} y-q)+p_\alpha\left(\mathcal{T}_n y-\ell y\right)\right\}\right\} \ \ \ \ \ \ \ \ \ \  \ \ \ \ \ \ \ \ \ \ (3.6)
\end{aligned}
$$

This implies that for every $n \in \mathbb{N}$,
$$
p_\alpha\left(\mathcal{T}_n x-\mathcal{T}_n y\right) \leq a p_\alpha(\ell x-\ell y)+(1-a) \max \left\{p_\alpha\left(\mathcal{T}_n x-\ell x\right), p_\alpha\left(\mathcal{T}_n y-\ell y\right)\right\} \ \ \ \ \ \ \ \ \ \  \ \ \ \ \ \ \ \ \ \ (3.7)
$$

for all $x, y \in \mathcal{M}$ and for all $p_\alpha \in \mathcal{A}^*(\tau)$.\\

Moreover, $\ell$ being nonexpansive on $\mathcal{M}$, implies that $\ell$ is $\|\cdot\|_{\mathscr{B}}$-nonexpansive and, hence, $\|\cdot\|_{\mathscr{B}}$-continuous. Since the norm topology on $\varepsilon_{\mathscr{B}}$ has a base of neighbourhoods of zero consisting of $\tau$-closed sets and $\mathcal{M}$ is $\tau$-sequentially complete, therefore, $\mathcal{M}$ is a $\|\cdot\|_{\mathscr{B}}$-sequentially complete subset of $\left(\varepsilon_{\mathcal{B}},\|\cdot\|_{\mathscr{B}}\right)$ (see proof in [10, Theorem 1.2]). Thus from \hyperlink{muc3.2}{\textcolor{Darkblue}{Theorem 3.2}}, for every $n \in \mathbb{N}, \mathcal{T}_n$ and $l$ have unique common fixed point $x_n$ in $\mathcal{M}$, i.e.,

$$
x_n=\mathcal{T}_n x_n=\ell x_n \ \ \ \ \ \ \ \ \ \  \ \ \ \ \ \ \ \ \ \ \ \ \ \ \ \ \ \ \ \ \ \ \ \ \ \ \ \ \ \ \ \ \ \ \ \ \ \ \ \  \ \ \ \ \ \ \ \ \ \ (3.8)
$$

for each $n \in \mathbb{N}$.\\
(i) As $\mathcal{M}$ is $\tau$-sequentially compact and $\left\{x_n\right\}$ is a sequence in $\mathcal{M}$, so $\left\{x_n\right\}$ has a convergent subsequence $\left\{x_m\right\}$ such that $x_m \rightarrow y \in \mathcal{M}$. As $L$ and $\mathcal{T}$ are continuous and\\
$$
x_m=\ell x_m=\mathcal{T}_m x_m=k_m \mathcal{T} x_m+\left(1-k_m\right) q
$$

so it follows that $y=\mathcal{T} y=\ell y$.\\
(ii) As $\mathcal{T}$ is compact and $\left\{x_n\right\}$ is bounded, so $\left\{\mathcal{T} x_n\right\}$ has a subsequence $\left\{\mathcal{T} x_m\right\}$ such that $\left\{\mathcal{T} x_m\right\} \rightarrow z \in \mathcal{M}$. Now we have\\
$$
x_m=\mathcal{T}_m x_m=k_{\mathrm{m}} \mathcal{T} x_m+\left(1-k_m\right) q
$$

Proceeding to the limit as $m \rightarrow \infty$ and using the continuity of $l$ and $\mathcal{T}$, we have $\ell z=z=\mathcal{T} z$.\\
(iii) The sequence $\left\{x_n\right\}$ has a subsequence $\left\{x_m\right\}$ converges to $u \in \mathcal{M}$. Since $\boldsymbol{l}$ is weakly continuous and so as in (i), we have $\iota u=u$. Now,

$$
x_m=l x_m=\mathcal{T}_m x_m=k_m \mathcal{T} x_m+\left(1-k_m\right) q
$$

implies that\\

$$
\ell x_m-\mathcal{T} x_m=\left(1-k_m\right)\left[q-\mathcal{T} x_m\right] \rightarrow 0
$$

as $m \rightarrow \infty$. The demiclosedness of $\ell-\mathcal{T}$ at 0 implies that $(\ell-\mathcal{T}) u=0$. Hence $\ell u=u=\mathcal{T} u$. This completes the proof.\\

\textbf{Example 3.5.} Let $\mathcal{X}=\mathbb{R}^2$ and $\mathcal{M}=\left\{0,1,1-\dfrac{1}{n-1}: n \in \mathbb{N}\right\}$ be endowed with usual metric. Define $\mathcal{T} 1=0$ and $\mathcal{T} 0=\mathcal{T}\left(1-\dfrac{1}{n-1}\right)=1$ for all $n \in \mathbb{N}$. Clearly, $\mathcal{M}$ is not convex. Let $\ell x=x$ for all $x \in \mathcal{M}$. Now $\mathcal{T}$ and $\ell$ satisfy \hyperlink{dau3.2}{\textcolor{Darkblue}{(3.2)}} together with all other conditions of \hyperlink{muc3.4}{\textcolor{Darkblue}{Theorem 3.4}}$(\mathrm{i})$ except the condition that $\mathcal{T}$ is continuous. Note that $\mathcal{F}(\mathcal{T}) \cap \mathcal{F}(\ell)=\emptyset$.\\

\textbf{Example 3.6.} Let $\mathcal{X}=\mathbb{R}^2$ be endowed with the norm defined by $\|(a, b)\|=|a|+|b|,(a, b) \in \mathbb{R}^2$.\\
(1) Let $\mathcal{M}=\mathcal{A} \cup \mathscr{B}$, where $\mathcal{A}=\{(a, b) \in \mathcal{X}: 0 \leq a \leq 1,0 \leq b \leq 4\}$ and $\mathscr{B}=\{(a, b) \in \mathcal{X}: 2 \leq a \leq 3,0 \leq b \leq 4\}$.

Define $\mathcal{T}: \mathcal{M} \rightarrow \mathcal{M}$ by

$$
\mathcal{T}(a, b)= \begin{cases}(2, b) & \text { if }(a, b) \in \mathscr{A} \\ (1, b) & \text { if }(a, b) \in \mathscr{B}\end{cases}
$$


and $\mathcal{L}(x)=x$ for all $x \in \mathcal{M}$. All the conditions of Theorem 3.4 (ii) are satisfied except that $\mathcal{M}$ is not convex. Note that $\mathcal{F}(\mathcal{T}) \cap \mathcal{F}(l)=\emptyset$.\\

(2) $\mathcal{M}=\{(a, b) \in \mathcal{X}: 2 \leq a<\infty, 0 \leq b \leq 1\}$ and $\mathcal{T}: \mathcal{M} \rightarrow \mathcal{M}$ is defined by

$$
\mathcal{T}(a, b)=\{(a+1, b):(a, b) \in \mathcal{M}\}
$$


Define $\ell(x)=x$ for all $x \in \mathcal{M}$. All the conditions of \hyperlink{muc3.4}{\textcolor{Darkblue}{Theorem 3.4}}$($ ii $)$ are satisfied except that $\mathcal{T}(\mathcal{M})$ is compact. Note $\mathcal{F}(\mathcal{T}) \cap \mathcal{F}(\ell)=\emptyset$. Notice that $\mathcal{M}$, being convex and $\mathcal{T}$-invariant.\\

(3) If $\mathcal{M}=\{(a, b) \in \mathcal{X}: 0 \leq a<1,0 \leq b \leq 1\}$ and $\mathcal{T}: \mathcal{M} \rightarrow \mathcal{M}$ is defined by

$$
\mathcal{T}(a, b)=\left(\frac{a}{2}, \frac{b}{3}\right) \quad \text { and } \quad L(x)=x \quad \text { for all } x \in \mathcal{M}
$$


All of the conditions of \hyperlink{muc3.4}{\textcolor{Darkblue}{Theorem 3.4}} (ii) are satisfied except the fact that $\mathcal{M}$ is closed. However $\mathcal{F}(\mathcal{T}) \cap \mathcal{F}(\boldsymbol{l})=\emptyset$.\\

\textbf{Example 3.7.} Let $\mathcal{M}=\mathbb{R}^2$ be endowed with the norm defined by $\|(a, b)\|=|a|+|b|,(a, b) \in \mathbb{R}^2$. Define $\mathcal{T}$ and $\mathcal{L}$ on $\mathcal{M}$ as follows:

$$
\begin{aligned}
& \mathcal{T}(x, y)=\left(\dfrac{1}{2}(x-2), \dfrac{1}{2}\left(x^2+y-4\right)\right) \\
& \mathscr{L}(x, y)=\left(\frac{1}{2}(x-2),\left(x^2+y-4\right)\right)
\end{aligned}
$$


Obviously, $\mathcal{T}$ is $\ell$-nonexpansive but $\ell$ is not linear. Moreover, $\mathcal{F}(\mathcal{T})=\{-2,0\},\\
\mathcal{F}(l)=\{(-2, y): y \in \mathbb{R}\}$ and the set of coincidence points of $\ell$ and $\mathcal{T}$, that is\\ $\mathcal{C}(\ell, \mathcal{T})=\left\{(x, y): y=4-x^2, x \in \mathbb{R}\right\}$. Thus $(\mathcal{T}, \ell)$ is a continuous, which is not compatible pair, and $(-2,0)$ is a common fixed point of $L$ and $\mathcal{T}$.\\

An application of Theorem 3.4, we prove the following more general result in best approximation theory.\\

\textbf{Theorem 3.8.} Let $\mathcal{T}$ and $l$ be self-maps of $a$ Hausdorff locally convex space $(\varepsilon, \tau)$ and $\mathcal{M}$ a subset of $\varepsilon$ such that $\mathcal{T}(\partial \mathcal{M}) \subseteq \mathcal{M}$, where $\partial \mathcal{M}$ stands for the boundary of $\mathcal{M}$ and $x_0 \in \mathcal{F}(\mathcal{T}) \cap \mathcal{F}(l)$. Suppose that $\ell$ is nonexpansive and linear on $\mathcal{D}_a$. Further, suppose $\mathcal{T}$ and $\ell$ satisfy \hyperlink{dau3.2}{\textcolor{Darkblue}{(3.2)}} for all $x, y \in \mathscr{D}_a^{\prime}=\mathscr{D}_a \cup\left\{x_0\right\}$ and pair $(\mathcal{T}, \boldsymbol{\ell})$ are subcompatible on $\mathscr{D}_a$. If $\mathscr{D}_a$ is nonempty convex and $\ell\left(\mathscr{D}_a\right)=\mathscr{D}_a$, then $\mathcal{T}$ and I have a common fixed point in $\mathscr{D}_a$ provided one of the following conditions holds:\\
(i) $\mathscr{D}_a$ is $\tau$-sequentially compact;\\
(ii) $\mathcal{T}$ is a compact map;\\
(iii) $\mathscr{D}_a$ is weakly compact in $(\mathcal{E}, \tau), \Omega$ is weakly continuous and $\ell-\mathcal{T}$ is demiclosed at 0 .\\

\textbf{Proof.} First, we show that $\mathcal{T}$ is self-maps on $\mathscr{D}_a$, i.e., $\mathcal{T}: \mathscr{D}_a \rightarrow \mathscr{D}_a$. Let $y \in \mathscr{D}_a$, then $\ell y \in \mathscr{D}_a$, since $\ell\left(\mathscr{D}_a\right)=\mathscr{D}_a$. Also, if $y \in \partial \mathcal{M}$, then $\mathcal{T} y \in \mathcal{M}$, since $\mathcal{T}(\partial \mathcal{M}) \subseteq \mathcal{M}$. Now since $\mathcal{T} x_0=x_0=\ell x_0$, so for each $p_\alpha \in \mathcal{A}^*(\tau)$, we have from \hyperlink{dau3.2}{\textcolor{Darkblue}{(3.2)}}

$$
\begin{aligned}
p_\alpha\left(\mathcal{T} y-x_0\right) & =p_\alpha\left(\mathcal{T} y-\mathcal{T} x_0\right) \\
& \leq a p_\alpha\left(\ell y-\ell x_0\right)+(1-a) \max \left\{p_\alpha(\mathcal{T} y-\ell y), p_\alpha\left(\mathcal{T} x_0-\ell x_0\right)\right\} \\
& \leq a p_\alpha\left(\ell y-x_0\right)+(1-a) \max \left\{p_\alpha\left(\mathcal{T} y-x_0\right)+p_\alpha\left(\ell y-x_0\right)\right\} \\
& =p_\alpha\left(\ell y-x_0\right)+(1-a) p_\alpha\left(\mathcal{T} y-x_0\right)
\end{aligned}
$$


So, we have\\
$$
a p_\alpha\left(\mathcal{T} y-\mathcal{T} x_0\right) \leq p_\alpha\left(\ell y-x_0\right)
$$


Now, $\mathcal{T} y \in \mathcal{M}$ and $\ell y \in \mathscr{D}_a$, this implies that $\mathcal{T} y$ is also closest to $x_0$, so $\mathcal{T} y \in \mathscr{D}_a$. Consequently $\mathcal{T}$ and $\ell$ are self-maps on $\mathscr{D}_a$. The conditions of \hyperlink{muc3.4}{\textcolor{Darkblue}{Theorem 3.4}}$\left((\mathrm{i})\right.$-(iii)) are satisfied and, hence, there exists a $v \in \mathscr{D}_a$ such that $\mathcal{T} v=v=\ell \nu$. This completes the proof.















































\vspace{2cm}
\printbibliography


\end{document}